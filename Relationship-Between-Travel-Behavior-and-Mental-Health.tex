% Options for packages loaded elsewhere
\PassOptionsToPackage{unicode}{hyperref}
\PassOptionsToPackage{hyphens}{url}
\PassOptionsToPackage{dvipsnames,svgnames,x11names}{xcolor}
%
\documentclass[
  letterpaper,
  authoryear]{elsarticle}

\usepackage{amsmath,amssymb}
\usepackage{iftex}
\ifPDFTeX
  \usepackage[T1]{fontenc}
  \usepackage[utf8]{inputenc}
  \usepackage{textcomp} % provide euro and other symbols
\else % if luatex or xetex
  \usepackage{unicode-math}
  \defaultfontfeatures{Scale=MatchLowercase}
  \defaultfontfeatures[\rmfamily]{Ligatures=TeX,Scale=1}
\fi
\usepackage{lmodern}
\ifPDFTeX\else  
    % xetex/luatex font selection
\fi
% Use upquote if available, for straight quotes in verbatim environments
\IfFileExists{upquote.sty}{\usepackage{upquote}}{}
\IfFileExists{microtype.sty}{% use microtype if available
  \usepackage[]{microtype}
  \UseMicrotypeSet[protrusion]{basicmath} % disable protrusion for tt fonts
}{}
\makeatletter
\@ifundefined{KOMAClassName}{% if non-KOMA class
  \IfFileExists{parskip.sty}{%
    \usepackage{parskip}
  }{% else
    \setlength{\parindent}{0pt}
    \setlength{\parskip}{6pt plus 2pt minus 1pt}}
}{% if KOMA class
  \KOMAoptions{parskip=half}}
\makeatother
\usepackage{xcolor}
\setlength{\emergencystretch}{3em} % prevent overfull lines
\setcounter{secnumdepth}{5}
% Make \paragraph and \subparagraph free-standing
\ifx\paragraph\undefined\else
  \let\oldparagraph\paragraph
  \renewcommand{\paragraph}[1]{\oldparagraph{#1}\mbox{}}
\fi
\ifx\subparagraph\undefined\else
  \let\oldsubparagraph\subparagraph
  \renewcommand{\subparagraph}[1]{\oldsubparagraph{#1}\mbox{}}
\fi


\providecommand{\tightlist}{%
  \setlength{\itemsep}{0pt}\setlength{\parskip}{0pt}}\usepackage{longtable,booktabs,array}
\usepackage{calc} % for calculating minipage widths
% Correct order of tables after \paragraph or \subparagraph
\usepackage{etoolbox}
\makeatletter
\patchcmd\longtable{\par}{\if@noskipsec\mbox{}\fi\par}{}{}
\makeatother
% Allow footnotes in longtable head/foot
\IfFileExists{footnotehyper.sty}{\usepackage{footnotehyper}}{\usepackage{footnote}}
\makesavenoteenv{longtable}
\usepackage{graphicx}
\makeatletter
\def\maxwidth{\ifdim\Gin@nat@width>\linewidth\linewidth\else\Gin@nat@width\fi}
\def\maxheight{\ifdim\Gin@nat@height>\textheight\textheight\else\Gin@nat@height\fi}
\makeatother
% Scale images if necessary, so that they will not overflow the page
% margins by default, and it is still possible to overwrite the defaults
% using explicit options in \includegraphics[width, height, ...]{}
\setkeys{Gin}{width=\maxwidth,height=\maxheight,keepaspectratio}
% Set default figure placement to htbp
\makeatletter
\def\fps@figure{htbp}
\makeatother

% These are extra latex packages that the document depends on
% 
\usepackage{siunitx}
\makeatletter
\makeatother
\makeatletter
\@ifpackageloaded{bookmark}{}{\usepackage{bookmark}}
\makeatother
\makeatletter
\@ifpackageloaded{caption}{}{\usepackage{caption}}
\AtBeginDocument{%
\ifdefined\contentsname
  \renewcommand*\contentsname{Table of contents}
\else
  \newcommand\contentsname{Table of contents}
\fi
\ifdefined\listfigurename
  \renewcommand*\listfigurename{List of Figures}
\else
  \newcommand\listfigurename{List of Figures}
\fi
\ifdefined\listtablename
  \renewcommand*\listtablename{List of Tables}
\else
  \newcommand\listtablename{List of Tables}
\fi
\ifdefined\figurename
  \renewcommand*\figurename{Figure}
\else
  \newcommand\figurename{Figure}
\fi
\ifdefined\tablename
  \renewcommand*\tablename{Table}
\else
  \newcommand\tablename{Table}
\fi
}
\@ifpackageloaded{float}{}{\usepackage{float}}
\floatstyle{ruled}
\@ifundefined{c@chapter}{\newfloat{codelisting}{h}{lop}}{\newfloat{codelisting}{h}{lop}[chapter]}
\floatname{codelisting}{Listing}
\newcommand*\listoflistings{\listof{codelisting}{List of Listings}}
\makeatother
\makeatletter
\@ifpackageloaded{caption}{}{\usepackage{caption}}
\@ifpackageloaded{subcaption}{}{\usepackage{subcaption}}
\makeatother
\makeatletter
\@ifpackageloaded{tcolorbox}{}{\usepackage[skins,breakable]{tcolorbox}}
\makeatother
\makeatletter
\@ifundefined{shadecolor}{\definecolor{shadecolor}{rgb}{.97, .97, .97}}
\makeatother
\makeatletter
\makeatother
\makeatletter
\makeatother
\journal{Transportation Research Part A}
\ifLuaTeX
  \usepackage{selnolig}  % disable illegal ligatures
\fi
\usepackage[]{natbib}
\bibliographystyle{elsarticle-harv}
\IfFileExists{bookmark.sty}{\usepackage{bookmark}}{\usepackage{hyperref}}
\IfFileExists{xurl.sty}{\usepackage{xurl}}{} % add URL line breaks if available
\urlstyle{same} % disable monospaced font for URLs
\hypersetup{
  pdftitle={Relationship Between Travel Behavior and Mental Health},
  pdfauthor={Gregory S. Macfarlane; Emily K. Blanchard},
  colorlinks=true,
  linkcolor={blue},
  filecolor={Maroon},
  citecolor={Blue},
  urlcolor={Blue},
  pdfcreator={LaTeX via pandoc}}

\setlength{\parindent}{6pt}
\begin{document}

\begin{frontmatter}
\title{Relationship Between Travel Behavior and Mental Health}
\author[1]{Gregory S. Macfarlane%
%
}
 \ead{gregmacfarlane@gmail.com} 
\author[1]{Emily K. Blanchard%
%
}
 \ead{ekblnchrd@gmail.com} 

\affiliation[1]{organization={Civil and Construction Engineering
Department, Brigham Young University},addressline={430
EB},city={Provo},postcode={84602},postcodesep={}}

\cortext[cor1]{Corresponding author}


        
\begin{abstract}
The abstract is a crucial component of any scientific paper, as it
provides a summary of the research and its main findings. This paper
provides guidelines for writing an effective scientific abstract. The
first step is to identify the key elements of the research, such as the
research question, methods, results, and conclusions. Next, the abstract
should be written in a clear and concise manner, using simple language
and avoiding technical jargon. The abstract should also be structured,
with a clear introduction, methods section, results section, and
conclusion. Additionally, the abstract should accurately and succinctly
convey the main findings of the research, highlighting the significance
and implications of the work. By following these guidelines, researchers
can ensure that their abstract effectively communicates the key aspects
of their research and attracts the attention of potential readers. -
Written by ChatGPT
\end{abstract}





\end{frontmatter}
    \ifdefined\Shaded\renewenvironment{Shaded}{\begin{tcolorbox}[borderline west={3pt}{0pt}{shadecolor}, interior hidden, breakable, frame hidden, sharp corners, enhanced, boxrule=0pt]}{\end{tcolorbox}}\fi

\bookmarksetup{startatroot}

\hypertarget{preface}{%
\section*{Preface}\label{preface}}
\addcontentsline{toc}{section}{Preface}

\markboth{Preface}{Preface}

This is a template repository that I and my students can use to start
projects that will implement the workflow presented in my
\href{https://gregmacfarlane.github.io/lab/workflow.html}{lab
documentation}. It also serves as an instruction manual in this
workflow, a template article, and a sandbox for me to practice and
learn. I encourage students to use the
\href{https://quarto.org/docs/guide/}{Quarto Guide} as their primary
reference.

The document in this template renders to two\footnote{I hope to make it
  possible to render the article to a BYU Engineering thesis as well.
  Give me a bit of time.} outputs:

\begin{itemize}
\item
  A website
\item
  An Elsevier journal article
\end{itemize}

To render this document, use the command \texttt{quarto\ render} in your
terminal pointed at the working directory. This will create a website
available locally in a \texttt{\_book} folder and a PDF of the article
stored in that folder.

To render your website \emph{and} push its content to a live website,
use the command \texttt{quarto\ publish\ gh-pages}. Details of this
process are available on the
\href{https://quarto.org/docs/publishing/github-pages.html\#publish-command}{Quarto
guide}.

You can change the article to a different publisher by following the
directions at the \href{https://github.com/quarto-journals}{Quarto
Journal Templates GitHub} repository.

\bookmarksetup{startatroot}

\hypertarget{introduction}{%
\section{Introduction}\label{introduction}}

The introduction of your report is not simply an ``introduction'', but
rather a \textbf{motivation} of why your project matters. What is the
cost of not solving this problem? What have been previous attempts to
solve this problem? The \emph{why} is more important than the
\emph{what}. Why is this article worthy of archiving?

The introduction to an article is usually three or so paragraphs long.
First, state the overall context of the problem, with citations to basic
statistics and previous major findings. Then, Discuss the specific
context of your research; why have previous methods or research not
addressed your specific issue? You also need to have citations here.

In the third paragraph, you can introduce the purpose of your research
and the new method or data you bring to the issue. This paragraph often
begins with a statement similar to ``In this paper, we present\ldots{}''

An outline of the article is then usually a good idea, even though the
outline might change little from article to article. A Literature Review
details prior research in this area, and discusses its strengths and
limitations. A Methodology section identifies the data, models, and
other elements you use, while a Results section presents the outcomes of
your methodology with tables and figures and a comprehensive discussion
of their meanings. Discussions, and Conclusions sections may present
your interpretations of your findings for future research, policy, etc.

\bookmarksetup{startatroot}

\hypertarget{literature-review}{%
\section{Literature Review}\label{literature-review}}

\textbf{Mental health is a crucial aspect of overall well-being and
mental health is decreasing (as urbanization is increasing, as a result
of COVID-19).}

Mental health is a complex issue that can be influenced by various
determinants, including social, economic, and environmental factors.
Addressing the determinants of positive mental health is crucial in
promoting well-being \citep{barry2009} . The relationship between
natural environments and positive mental health has been a topic of
interest for researchers for decades and has worked towards addressing
the determinants of positive mental health. Barry reviews the
determinants of positive mental health, emphasizing the importance of
social support and social connectedness, a sense of control and
autonomy, self-esteem, and meaning and purpose in life. She argues that
mental health promotion should focus on building resilience, enhancing
positive emotions, and developing skills and competencies to manage
adversity \citep{barry2009} . \citet{white2021} adds that poor mental
health creates more of a challenge for those in high-income countries
which may be a consequence of urbanization and lack of natural
environment \citep{white2021}.

During the COVID-19 pandemic, access to natural environments became more
limited for many people due to lock downs and social distancing
measures. However, a study conducted by \citet{pouso2021} found that
contact with blue-green spaces during the pandemic lock down was
beneficial for mental health. The study was conducted in Spain and
surveyed individuals who had access to blue-green spaces, such as
coastal areas or parks, during the lock down. The results showed that
those who had contact with these spaces reported lower levels of stress
and anxiety. Similarly, additional research supports the idea that
access to natural environments can contribute to positive mental health
outcomes, independent of the COVID-19 pandemic. A study conducted by
\citet{white2021} investigated the associations between green and blue
spaces and mental health across 18 countries. The study found that
individuals who had access to more green and blue spaces reported higher
levels of mental health and well-being. Specifically, exposure to green
spaces was associated with lower levels of depression and anxiety, while
exposure to blue spaces was associated with improved mood and cognitive
function. Access to green and blue spaces could serve as a coping
strategy during stressful situations like pandemics.

Overall, the research suggests that access to natural environments can
have a positive impact on mental health and well-being. These findings
have important implications for mental health promotion and urban
planning. Providing individuals with access to green and blue spaces can
promote positive mental health outcomes, particularly during times of
stress and adversity such as the COVID-19 pandemic.

\textbf{Social isolation or social exclusion is affected by one's
ability to travel and in turn affects mental health.}

\citet{stanley2011} investigates the link between mobility, social
exclusion, and well-being. The authors conducted a survey among
residents of Melbourne, Australia, to analyze the impact of social
exclusion on well-being and how mobility can mediate this relationship.
The study found that mobility can improve well-being and reduce social
exclusion, which can have a positive impact on individual mental health.
Similarly, \citet{delbosc2011} explored the influence of transport
disadvantage and social exclusion on well-being. The authors used data
from a survey of individuals in Melbourne, Australia, to examine the
relationship between transport disadvantage and well-being, as well as
the relationship between social exclusion and well-being. The study
found that both transport disadvantage and social exclusion can
negatively affect well-being, with social exclusion having a stronger
impact. The ability and choice to make trips affects well-being and
mental health of individuals.

When looking specifically at the effects of the COVID-19 pandemic,
\citet{loades2020} , focused on the impact of social isolation and
loneliness on the mental health of children and adolescents. The authors
conducted a rapid systematic review of existing literature on the topic
and found that social isolation and loneliness can have a negative
impact on mental health, including increased symptoms of anxiety,
depression, and post-traumatic stress disorder.

Overall, these articles suggest that social exclusion and isolation,
transport disadvantage, and loneliness can negatively impact individual
well-being and mental health. On the other hand, mobility and contact
with others can have a positive impact on individual well-being. These
findings highlight the importance of social support networks, access to
transportation, and the need to address social exclusion and isolation
to promote individual well-being and mental health.

\textbf{The environment in which people live and interact affects their
mental health.} (built environment)

In recent years, a growing body of research has explored the
relationship between the built environment and mental health. Several
studies have explored the association between the built environment and
mental health \citep{pelgrims2021}, \citep{white2021},
\citep{pouso2021}, \citep{rautio2018}, \citep{gascon2018},
\citep{engemann2019}, \citep{hoisington2019}. The findings of these
studies specifically investigate the association between the urban
environment, green and blue spaces, and mental health.

\citet{pelgrims2021} conducted a study in Brussels, Belgium, to
investigate the association between the urban environment and mental
health. The study found that exposure to noise pollution, air pollution,
and lack of green space were associated with higher levels of depressive
symptoms and stress. In contrast, living close to green space was
associated with lower levels of stress and depressive symptoms.
Similarly, \citet{white2021} conducted a cross-country study in 18
countries to investigate the relationship between green and blue spaces
and mental health. The study found that exposure to green and blue
spaces was associated with better mental health outcomes, including
lower levels of stress, anxiety, and depression. In addition,
\citet{pouso2021} investigated the impact of contact with blue-green
spaces during the COVID-19 pandemic lock down on mental health. The
study found that individuals who had more contact with blue-green spaces
reported better mental health outcomes, including lower levels of stress
and anxiety.

\citet{rautio2018} conducted a systematic review of the literature on
the relationship between the living environment and depressive mood. The
review found that several environmental factors, including noise
pollution, air pollution, and lack of green space, were associated with
higher levels of depressive symptoms. The review also found that access
to green spaces, such as parks and gardens, was associated with lower
levels of depressive symptoms. In addition to the literature review by
\citet{rautio2018} , \citet{gascon2018} conducted a cross-sectional
study to investigate the relationship between long-term exposure to
residential green and blue spaces and anxiety and depression in adults.
The study found that exposure to green and blue spaces was associated
with lower levels of anxiety and depression in adults.

\citet{engemann2019} conducted a study to investigate the relationship
between residential green space in childhood and the risk of psychiatric
disorders from adolescence into adulthood. The study found that
individuals who grew up in areas with more residential green space had a
lower risk of developing psychiatric disorders. This may indicate that
having access to green space can affect one's mental health in the
future.

Overall, the literature suggests that exposure to green and blue spaces
is associated with better mental health outcomes, while exposure to
noise and air pollution is associated with higher levels of stress,
anxiety, and depression. \citet{hoisington2019} reviewed the literature
on the relationship between the built environment and mental health and
identified ten questions concerning this relationship. Their review
highlighted the need for further research to understand the complex
interactions between the built environment and mental health outcomes.

\textbf{Types of activities or travel affect people's mental health
differently.}

Travel behavior has been recognized as a significant factor influencing
social and mental health among individuals. The following discussion
aims to explore the relationship between travel behavior and mental
well-being outcomes by analyzing findings from four relevant studies.

\citet{syahputri2022} conducted a study in the Bandung Metropolitan area
to investigate the effect of travel satisfaction and activity-travel
patterns of other household members on social and mental health. This
model showed that those who stay at home to work, or study tend to
experience high travel satisfaction and in turn high social and mental
health. These people appreciate the time that they spend outside of the
home, which results in an increase in their social and mental health.
However, if people spend too much time at home studying or working, then
they typically experience lower mental health because they are unable to
interact with others outside of the home. But when these people have the
opportunity to leave the home to travel, they experience better travel
satisfaction and in turn better social and mental health. Similarly,
\citet{friman2017} investigated how travel affects emotional well-being
and life satisfaction. They pointed out that those who regularly
traveled to work experienced less life satisfaction than those who
worked from home. However, the study revealed a positive association
between travel and emotional well-being, indicating that travel
experiences contribute to improved mood and overall life satisfaction.

When exploring the relationship between gender, mental health, and
travel, \citet{mackett2022} conducted a comprehensive study. The study
found that women tend to experience higher levels of mental health
issues related to travel, such as anxiety and stress, due to various
factors, including route finding, safety concerns, and interacting with
others. In another study \citet{mackett2021} explored the relationship
between mental health and travel behavior. The study found that travel
behavior, such as active travel (e.g., walking or cycling) and public
transportation use, was associated with better mental health outcomes,
including reduced stress and improved well-being. \citet{mackett2021}
pointed at that the ``relationship between mental health and travel is
complex'' because mental health issues can affect one's ability to
travel and travel can worsen mental health issues.

The reviewed studies highlight the significant influence of travel
behavior on social and mental health outcomes. Factors such as travel
satisfaction, activity-travel patterns of household members, and gender
differences play crucial roles in shaping these outcomes. Positive
travel experiences are associated with improved emotional well-being and
life satisfaction. However, it is essential to consider the potential
challenges and disparities faced by different genders in relation to
mental health and travel.

\textbf{Recently, a group used mobile phone-based sensing to look at
daily activities and environmental exposures and the connection to
anxiety symptoms.}

The article by \citet{lan2022} investigates the relationship between
daily space-time activities, multiple environmental exposures, and
anxiety symptoms. The study utilizes a cross-sectional mobile
phone-based sensing approach to gather data. The goal was to understand
how individuals' activities in different environmental settings
contribute to their anxiety levels.

The study collected data from participants using mobile phone sensors,
including GPS and accelerometers, to track their spatial movements and
activity patterns. Anxiety symptoms were measured using a standardized
questionnaire. Additionally, multiple environmental exposures, such as
air pollution, noise, and green space availability were assessed.

The findings of the study indicate that certain space-time activities
and environmental exposures are associated with anxiety symptoms.
Participants who spent more time in areas with higher levels of air
pollution and noise reported higher anxiety levels. Conversely, spending
time in green spaces was associated with lower anxiety symptoms. The
study also found that physical activity and social interaction in
various environments were linked to decreased anxiety.

Overall, the study provides insights into the complex relationship
between daily activities, environmental exposures, and anxiety symptoms.
The findings highlight the importance of considering the spatial context
and environmental factors when studying mental health outcomes. The
results suggest that promoting access to green spaces, reducing exposure
to air pollution and noise, and encouraging physical activity and social
interaction in supportive environments could have positive impacts on
anxiety levels.

\bookmarksetup{startatroot}

\hypertarget{methodology}{%
\section{Methodology}\label{methodology}}

\bookmarksetup{startatroot}

\hypertarget{applications}{%
\section{Applications}\label{applications}}

\bookmarksetup{startatroot}

\hypertarget{conclusions}{%
\section{Conclusions}\label{conclusions}}

This section need not be overly long. You should address any limitations
of your results, such as dependence on underlying assumptions or
geographic scope. You should also provide a map for future research.

Finally, you should underline the contributions of this work and any
practical relevance.

\bookmarksetup{startatroot}

\hypertarget{references}{%
\section*{References}\label{references}}
\addcontentsline{toc}{section}{References}

\markboth{References}{References}

\renewcommand{\bibsection}{}
\bibliography{references.bib}




\end{document}
